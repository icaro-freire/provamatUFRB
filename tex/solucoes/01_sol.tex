%========================
% Solução da Questão 01
%========================
\begin{solution}
 Nessa questão (e na maioria das que se seguem nessa prova) organização é 
 fundamental, visto que os cálculos podem ser extensos.

 Todavia, não há nada além da aplicação das definições das funções complexas
 $ \sen{z} = \frac{e^{iz} -  e^{-iz}}{2i} $ e $ \cos{z} = \frac{e^{iz} + e^{-iz}}{2} $.
 
 Assim, chamando $ E = \frac{2 \sen{i}}{i \cos{i} + \sen{i}} $, precisamos, 
 inicialmente, encontrar os valores de $ \sen{(i)} $ e $ \cos{(i)} $.
 Então,
 
 \begin{itemize}
  \item 
   $
    \sen{(\textcolor{red}{i})} 
    = \frac{e^{i (\textcolor{red}{i})} - e^{-i (\textcolor{red}{i})}}{2i} 
    = \frac{e^{-1} - e}{2i}
    = \frac{e^{-1} - e}{2i} \cdot \frac{\textcolor{blue}{(-i)}}{\textcolor{blue}{(-1)}}
    = \frac{e - 1/e}{2} \cdot i
    = \frac{e^2 - 1}{2e} \cdot i
   $
  \item
   $
    \cos{(\textcolor{red}{i})}
    = \frac{e^{i(\textcolor{red}{i})} + e^{-i(\textcolor{red}{i})}}{2}
    = \frac{e^{-1} + e}{2}
    = \frac{e + 1/e}{2}
    = \frac{e^2 + 1}{2e}
   $
 \end{itemize}
 
 Com isso, podemos encontrar, rapidamente:
 \begin{itemize}
  \item 
   $
    2 \sen{i} 
    = 2 \left(\, \cdot \frac{e^2 - 1}{2e} \cdot i \,\right)
    = \frac{e^2 - 1}{e} \cdot i
   $
  \item
  $
   i \cos{(i)} + \sen{(i)} 
   = i \cdot \left(\, \frac{e^2 + 1}{2e} \,\right) + \left(\, \frac{e^2 - 1}{2e} \cdot i \,\right)
   = \frac{2 e^2}{2 e} \cdot i
   = ei
  $
 \end{itemize}
 
 Assim, podemos encontrar o valor da expressão:
 \[
  E = \frac{2 \sen{(i)}}{i \cos{(i)} + \sen{(i)}} 
    = 2 \sen{(i)} \cdot \frac{1}{i \cos{(i)} + \sen{(i)}}
    = \left(\, \frac{e^2 - 1}{e} \cdot i \,\right) \cdot \left(\, \frac{1}{ei} \,\right) 
    = \frac{e^2 - 1}{e^2}
    = \ovalbox{$ 1 - e^{-2} $}.
 \]
\end{solution}